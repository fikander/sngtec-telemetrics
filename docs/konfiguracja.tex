\documentclass[11pt]{article}
\usepackage[MeX]{polski}
\usepackage[utf8]{inputenc}
\usepackage{amsmath}
\usepackage{verbatim}

\title{Projekt telemetryczny \\ Opis konfiguracji}
\author{Marcin Fatyga \and Karol Polok \and Szymon Starzycki \and Piotr Szcześniak}

\begin{document}
\begin{titlepage}

\maketitle

\thispagestyle{empty}
\end{titlepage}

\tableofcontents

\newpage

\section{Wstęp}
Niniejszy dokument zawiera opis konfiguracji Modułu Telemetrycznego.

\section{Ogólnie o konfiguracji}
Plik konfigurujacy powinien znajdować się w tym samym katalogu co binarka i nazywać się config.xml.

Powinien on rozpoczynać sie tagiem: (\verb|<!DOCTYPE TelemetronConfig>|). Cała jego zawartość powinna znajdować się wewnątrz tagu (\verb|<telemetron>|); powinien on wyglądać mniej więcej tak:
\begin{verbatim}
<!DOCTYPE TelemetronConfig>
<telemetron>
        <!-- opis chmury -->
        <!-- opisy sensorow -->
</telemetron>
\end{verbatim}



\section{Konfiguracja chmury}

\subsection{Pachube}

Przykładowa konfiguracja dla chmury typu:
\begin{verbatim}
  <cloud type="pachube" 
         feedno="42213" 
         apikey="7p1iGBicjYRQS20cEDjFdr3SnTldEyq-kL2XabP6iXw" />
\end{verbatim}



\section{Konfiguracja sensorów}
Obsługujemy sensory typów: sng, modbus, mock.

Opis powinien wyglądać jak:
\begin{verbatim}
<device type="typ" name="nazwa">
<mappings klucz="wartosc"
          inny_klucz="inna_wartosc">
</device>
\end{verbatim}

Klucze i wartości zdefiniowane są przez konkretne rodzaje sensorów. Mock nie jest konfigurowany.


\subsection{Modbus}
Aby skorzystać z sensora tego typu, należy w konfiguracji podać jako typ \verb|modbus|
\subsubsection{Port}
Urządzenie Telemetryczne komunikuje się z sensorami za pomocą RS-485. W konfiguracji wymagane
jest podanie właściwej nazwy portu szeregowego.

\subsubsection{Komunikacja z chmurą}
Format wiadomości Message w obu kierunkach (do i od) między chmurą i Urządzeniem Telemetrycznym jest identyczny.
Pod wartością \verb|key| znajdują się 2 bity - adres urządzenia / nr funkcji. Wartość \verb|value| to z kolei dane
dla tej funkcji (w przypadku komunikacji od chmury do urządzenia) lub zwrócone przez tę funkcję (w przeciwnym kierunku).
Wszelkie wartości powinny być podawane w formacie szesnastkowym.


\subsection{SNG}
Aby skorzystać z sensora tego typu, należy w konfiguracji podać jako typ \verb|sng|
\subsubsection{CommServer}
Komunikacja pomiędzy Modułem Telemetrycznym a fizycznymi urządzeniami odbywa się przy pomocy
programu CommServer. Połączenie z CommServerem odbywa się przy pomocy protokołu TCP/IP.
W związku z tym podczas konfigurowania należy ustawić odpowiedni adres (zmienna \verb|address|)
i numer portu (\verb|port|) serwera.

\subsubsection{Przesyłanie do czujników}
Aby otrzymaną od chmury wiadomość o kluczu \verb|key| przesłać do fizycznego urządzenia,
należy ustawić odpowiadający jej adres rozgłoszeniowy (na którym nasłuchuje urządzenie), 
oraz typ ramki. Adresy w SNG są 3-bajtowe, bajty oddzialane są kropkami. 
Możliwe typy przesyłanych wartości to:\\
\verb|OnOff, Dimm, Time, Date, Temp, Value|

\subsubsection{Odbieranie od czujników}
Aby wiadomość otrzymaną od czujnika przesłać do chmury, należy ustawić adresy na których urządzenie telemetryczne
ma nasłuchiwać, oraz które wartości otrzymywane nas interesują. Do każdego takiego adresu i typu wartości
należy przypisać klucz z chmury.

\subsubsection{Przykład}
\begin{verbatim}
  <device type="sng" name="beta">
    <mappings address="192.168.1.1"
              port="8888"
               />
  </device>
\end{verbatim}


\section{Łączenie urządzeń w topologię}

\section{Logowanie}

Logi zapisywane są na zmianę do plików \verb|logs_1| i \verb|logs_2|. Po osiągnięciu limitu 1000 linii, obecny plik z logami jest zamykany. Następnie program otwiera drugi plik, czyści jego zawartość i kontynuuje tam zapis. 

%% Ponadto komunikaty debugu są wypisywane na standardowe wyjście błędów.


\end{document}

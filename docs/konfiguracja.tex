\documentclass[11pt]{article}
\usepackage[MeX]{polski}
\usepackage[utf8]{inputenc}
\usepackage{amsmath}
\usepackage{verbatim}

\title{Projekt telemetryczny \\ Opis konfiguracji}
\author{Marcin Fatyga \and Karol Polok \and Szymon Starzycki \and Piotr Szcześniak}

\begin{document}
\begin{titlepage}

\maketitle

\thispagestyle{empty}
\end{titlepage}

\tableofcontents

\newpage

\section{Wstęp}
Niniejszy dokument zawiera opis konfiguracji Modułu Telemetrycznego.

\section{Ogólnie o konfiguracji}

\section{Konfiguracja chmury}

\subsection{Pachube}

\section{Konfiguracja sensorów}

\subsection{Modbus}

\subsection{SNG}

\subsubsection{CommServer}
Komunikacja pomiędzy Modułem Telemetrycznym a fizycznymi urządzeniami odbywa się przy pomocy
programu CommServer. Połączenie z CommServerem odbywa się przy pomocy protokołu TCP/IP.
W związku z tym podczas konfigurowania należy ustawić odpowiedni adres (zmienna \verb|address|)
i numer portu (\verb|port|) serwera.

\subsubsection{Przesyłanie do czujników}
Aby otrzymaną od chmury wiadomość o kluczu \verb|key| przesłać do fizycznego urządzenia,
należy ustawić odpowiadający jej adres rozgłoszeniowy (na którym nasłuchuje urządzenie), 
oraz typ ramki. Adresy w SNG są 3-bajtowe, bajty oddzialane są kropkami. 
Możliwe typy przesyłanych wartości to:\\
\verb|OnOff, Dimm, Time, Date, Temp, Value|

\subsubsection{Odbieranie od czujników}
Aby wiadomość otrzymaną od czujnika przesłać do chmury, należy ustawić adresy na których urządzenie telemetryczne
ma nasłuchiwać, oraz które wartości otrzymywane nas interesują. Do każdego takiego adresu i typu wartości
należy przypisać klucz z chmury.

\subsubsection{Przykład}
\begin{verbatim}
  <device type="sng" name="beta">
    <mappings address="192.168.1.1"
              port="8888"
               />
  </device>
\end{verbatim}


\section{Łączenie urządzeń w topologię}

\end{document}

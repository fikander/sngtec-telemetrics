\documentclass[11pt]{article}
\usepackage[MeX]{polski}
\usepackage[utf8]{inputenc}
\usepackage{amsmath}
\usepackage{verbatim}

\title{Projekt telemetryczny \\ Opis konfiguracji}
\author{Marcin Fatyga \and Karol Polok \and Szymon Starzycki \and Piotr Szcześniak}

\begin{document}
\begin{titlepage}

\maketitle

\thispagestyle{empty}
\end{titlepage}

\tableofcontents

\newpage

\section{Wstęp}
Niniejszy dokument zawiera opis konfiguracji Modułu Telemetrycznego.

\section{Ogólnie o konfiguracji}
Plik konfigurujacy powinien znajdować się w tym samym katalogu co binarka i nazywać się config.xml.

Powinien on rozpoczynać sie tagiem: (\verb|<!DOCTYPE TelemetronConfig>|). Cała jego zawartość powinna znajdować się wewnątrz tagu (\verb|<telemetron>|); powinien on wyglądać mniej więcej tak:
\begin{verbatim}
<!DOCTYPE TelemetronConfig>
<telemetron>
        <!-- opis chmury -->
        <!-- opisy sensorow -->
</telemetron>
\end{verbatim}


\section{Konfiguracja chmury}

\subsection{Pachube}

Aby urządzenie działało z serwisem pachube należy najpierw skonfigurować urządzenie aby używało chmury pachube (ustawiając pole \verb|type|).
Do konfiguracji należy podać:
\begin{enumerate}
\item Numer feedu do wysyłania. Można go stworzyć po zalogowaniu na stronie pachube za pomocą “Create feed”. Zaleca się tworzenie feedów prywatnych.
\item Numer feedu do odbierania danych.
\item Api key wygenerowany na stronie pachube mający dostęp do obu feedów (jeśli zostały stworzone jako feedy prywatne).
\item Konfiguracje mapowania kluczy datastreamów
\end{enumerate}

\subsubsection{Datastreamy}
Każdy feed zawiera datastreamy mogące zawierać dowolne dane (liczby całkowite, ułamkowe, oraz napisy). Każdy datastream jest identyfikowalny za pomocą unikalnego klucza (napisu). Urządzenie odbiera dane z feedu tłumacząc unikalny klucz przy użyciu mappera. Konfiguracja mappera pozwala na rozstrzygnięcie, jak klucze dostępne w serwisie pachube, mają być tłumaczone na klucze wysyłane do konkretnych urządzeń. Chmura pachube odbiera dane z feedów, następnie tłumaczy klucze w skonfigurowany sposób na klucze wysyłane do konkretnych sensorów.
Dane odbierane od sensorów i wysyłane do strony pachube również tłumaczone są przy pomocy mappera.

\subsubsection{Przykład}

Przykładowa konfiguracja dla chmury typu:
\begin{verbatim}
  <cloud type="pachube" 
         sendFeed="42213" 
         receiveFeed="42213" 
         apikey="7p1iGBicjYRQS20cEDjFdr3SnTldEyq-kL2XabP6iXw" />
\end{verbatim}




\section{Konfiguracja sensorów}
Obsługujemy sensory typów: sng, modbus, mock.\\

Opis powinien wyglądać tak:
\begin{verbatim}
<device type="typ" name="nazwa">
<mappings klucz="wartosc"
          inny_klucz="inna_wartosc">
</device>
\end{verbatim}

Klucze i wartości zdefiniowane są przez konkretne rodzaje sensorów. Mock nie jest konfigurowany.
W kluczach i wartościach podwójne podkreślenie \verb|__| jest używane jako separator i nie
powinno być używane w innych celach. Format XML nie obsługuje atrybutów zaczynających się od cyfry,
należy więc go poprzedzić separatorem. W szczególności niepoprawny jest wpis: \verb|1.1.1="wartosc"|,
poprawna wersja to: \verb|__1.1.1="wartosc"|


\subsection{Modbus}
Aby skorzystać z sensora tego typu, należy w konfiguracji podać jako typ \verb|modbus|
\subsubsection{Port}
Urządzenie Telemetryczne komunikuje się z sensorami za pomocą RS-485. W konfiguracji wymagane
jest podanie właściwej nazwy portu szeregowego.

\subsubsection{Komunikacja z chmurą}
Format wiadomości Message w obu kierunkach (do i od) między chmurą i Urządzeniem Telemetrycznym jest identyczny.
Pod wartością \verb|key| znajdują się 2 bity - adres urządzenia / nr funkcji. Wartość \verb|value| to z kolei dane
dla tej funkcji (w przypadku komunikacji od chmury do urządzenia) lub zwrócone przez tę funkcję (w przeciwnym kierunku).
Wszelkie wartości powinny być podawane w formacie szesnastkowym.

\subsubsection{Przykład}
Część pliku konfiguracyjnego dotycząca Modbusa powinna wyglądać tak:
\begin{verbatim}
  <device type="modbus" name="beta">
    <mappings port="/dev/ttyS0" />
  </device>
\end{verbatim}


\subsection{SNG}
Aby skorzystać z sensora tego typu, należy w konfiguracji podać jako typ \verb|sng|
\subsubsection{CommServer}
Komunikacja pomiędzy Modułem Telemetrycznym a fizycznymi urządzeniami odbywa się przy pomocy
programu CommServer. Połączenie z CommServerem odbywa się przy pomocy protokołu TCP/IP.
W związku z tym podczas konfigurowania należy ustawić odpowiedni adres (zmienna \verb|address|)
i numer portu (\verb|port|) serwera. Należy ponadto podać fizyczny adres urządzenia w sieci 
SNG (szczegóły niżej) jako parametr \verb|physicalAddress|.

\subsubsection{Przesyłanie do czujników}
Aby otrzymaną od chmury wiadomość o pewnym kluczu przesłać do fizycznego urządzenia,
należy ustawić odpowiadający jej adres rozgłoszeniowy (na którym nasłuchuje urządzenie), 
oraz typ ramki. Adresy w SNG są 3-bajtowe, bajty oddzialane są kropkami. 
Możliwe typy przesyłanych wartości to:\\
\verb|OnOff, Dimm, Time, Date, Temp, Value|\\
Wartości te należy oddzielić separatorem \verb|__| (2 podkreślenia).\\
\\
Przykładowo chcąc aby wartość o kluczu \verb|sciemnianie| była przesyłana na adres
grupowy \verb|1.2.3| i traktowana jako typ \verb|Dimm| należy ustawić:\\
\verb|sciemnianie="1.2.3__Dimm"|\\
\\
Natomiast jeśli klucz \verb|steruj_swiatlem| ma odpowiadać za sterowanie poziomem oświetlenia,
za te funkcje odpowiada adres grupowy \verb|2.3.4|, a wartość oczekiwana jest wartością typy
\verb|Value| to w konfiguracji powinno się znaleźć:\\
\verb|steruj_swiatlem="2.3.4__Value"|\\


\subsubsection{Odbieranie od czujników}
Aby wiadomość otrzymaną od czujnika przesłać do chmury, należy ustawić adres urządzenia od którego odbieramy,
adres na których urządzenie telemetryczne ma nasłuchiwać, oraz które wartości otrzymywane nas interesują i 
przypisać do nich klucz z chmury. Istnieje możliwość nasłuchiwania tylko na konkretnym adresie grupowym,
bez uwzględniania adresu fizycznego nadawcy. Należy w tym celu w konfiguracji jako adres fizyczny
podać \verb|0.0.0|.\\
\\
Chcąc monitorować temperaturę z wielu czujników, które wysyłają je na adres grupowy \verb|3.4.5| i wysyłać chmurze
jako wiadomość \verb|nowa_temp|, w konfiguracji powinien znaleźć się wpis:\\
\verb|__0.0.0__3.4.5__Temp="nowa_temp"|\\
\\
Aby móc odbierać sygnał o udanym włączeniu światła przez fizyczny obwód o adresie \verb|1.1.1|,
który rozgłasza tę informację na adres grupowy \verb|1.2.5| i przekazac jako wiadomosc 
\verb|swiatlo_wlaczono| do chmury powinniśmy skonfigurować moduł następująco:\\
\verb|__1.1.1__1.2.5__OnOff="swiatlo_wlaczono"|\\
\\
Jeśli czujnik temperatury o adresie fizycznym \verb|1.3.4| został skonfigurowany tak, że po przekroczeniu
pewnej temperatury w pomieszczeniu rozsyła na adres grupowy \verb|1.0.0| informację o tej temperaturze,
my zaś chcemy to odebrać i przesłać do chmury informację o kluczu \verb|temp_podniesiona| to powinniśmy w
pliku konfiguracyjnym dodać wpis:\\
\verb|__1.3.4__1.0.0__Temp="temp_podniesiona"|  

\subsubsection{Przykład}
Przykładowa część pliku konfiguracyjnego dotycząca SNG może wyglądać tak:
\begin{verbatim}
  <device type="sng" name="beta">
    <mappings address="192.168.1.1"
              port="8888"
              physicalAddress="4.5.6"
              sciemnianie="1.2.3__Dimm"
              steruj_swiatlem="2.3.4__Value"
              __0.0.0__3.4.5__Temp="nowa_temp"
              __1.1.1__1.2.5__OnOff="swiatlo_wlaczono"
              __1.3.4__1.0.0__Temp="temp_podniesiona"
               />
  </device>
\end{verbatim}

\subsubsection{Format danych w SNG}
Poprawne formaty wartości, obsługiwane w protokole SNG to:
\begin{itemize}

\item OnOff: \verb|on|, \verb|off|
\item Dimm: \verb|inc|, \verb|endInc|, \verb|dec|, \verb|endDec|
\item Time: \verb|hh:mm:ss|
\item Date: \verb|dd.mm.yyyy|, przy czym rok jest liczony modulo 256 i w takiej też postaci jest odsyłany
\item Temp: \verb|[-]A.B| , gdzie A należy do 0-255, B należy do 0-9,  minus opcjonalnie
\item Value: \verb|0-255| 
\end{itemize}


\section{Łączenie urządzeń w topologię}
Urządzenia można łączyć w topologię gwiazdy przy użyciu sieci lokalnej. Sytuacja taka może mieć zastosowanie np w przypadku
gdy tylko jedno urządzenie posiada modem GPS. Urządzenie to nazwiemy urządzeniem \textit{master} do którego podłączamy urządzenie typu
\textit{slave}. Do jednego urządzenia \textit{master} można podłączyć wiele urządzeń typu \textit{slave}, z których każde może być podłączone
tylko do jednego urządzenia \textit{master}.

\subsection{Konfiguracja urządzenia typu master}
W konfiguracji takiego urządzenia należy zadeklarować jeden z czujników jako sensor typu \verb|topology|. W konfiguracji
należy określi parametr \verb|port| - port na którym ma się odbywać komunikacja. Na tym porcie odbywa się nasłuchiwanie.
Przykładowy plik konfiguracyjny:
\begin{verbatim}
  <device type="topology" name="czujnik_topologia">
    <mappings port="8888"/>
  </device>
\end{verbatim}

\subsection{Konfiguracja urządzenia typu slave}
W konfiguracji takiego urządzenia należy zadeklarować chumre typu \verb|topology|. W konfiguracji
należy określi parametr \verb|port| - port na którym ma się odbywać komunikacja. Na tym porcie \textit{slave} próbuje połączyć
się z \textit{master}. Przykładowy plik konfiguracyjny:
\begin{verbatim}
  <cloud type="topology" 
         port="8888" />
\end{verbatim}


\section{Logowanie}

Logi zapisywane są na zmianę do plików \verb|logs_1| i \verb|logs_2|. Po osiągnięciu limitu 1000 linii, obecny plik z logami jest zamykany. Następnie program otwiera drugi plik, czyści jego zawartość i kontynuuje tam zapis. 

%% Ponadto komunikaty debugu są wypisywane na standardowe wyjście błędów.


\end{document}
